\documentclass[12pt, a4paper]{book}
\usepackage[utf8]{inputenc}
\usepackage[brazilian]{babel}
\usepackage{amsmath}
\usepackage{amsfonts}
\usepackage{amssymb}
\usepackage{verbatim}
\usepackage[T1]{fontenc}
\usepackage{accanthis}
\usepackage{graphicx}
\usepackage{physics}
\usepackage{siunitx}
\usepackage{cite}
\usepackage[top=2.5cm, bottom=2.5cm, left=3cm, right=3cm]{geometry}
\usepackage{makeidx}
\usepackage{helvet}
\usepackage{graphicx}
\usepackage{sectsty}
\usepackage{tocbibind}
\usepackage[brazilian,hyperpageref]{backref}	 % Paginas com as citações na bibl

\usepackage[alf]{abntex2cite}	% Citações padrão ABNT

%Criando Ambientes
\newtheorem{defi}{Definição}[subsection]
\newtheorem{teor}{Teorema}[subsection]
\newtheorem{lema}{Lema}[subsection]
\newtheorem{demo}{Demonstração}[subsection]
\newtheorem{nota}{Notação}[subsection]
\newtheorem{coro}{Corolário}[subsection]
\newtheorem{propo}{Proposição}[subsection]
\newtheorem{ex}{Exemplo}[subsection]
\newtheorem{sol}{Solução}[subsection]

\begin{document}
	\chapter{Soluções de uma Equação Diferencial}
	
	\section{Definição de Solução}
	
	\begin{defi}
		Qualquer função $f$ definida em algum intervalo $I$, que, quando substituído na ED, reduz a equação a uma identidade, é chamada de \textbf{Solução}. 
	\end{defi}

	\begin{ex}
		Dada a função $y(x) = c_{1} sen(2x) + c_{2} cos(2x)$, sendo $c_{1}$ e $c_{2}$ constantes arbitrárias, mostre que $y(x)$ é solução da EDO \ $y''+4y = 0$.
	\end{ex} 
	
	\begin{sol}
		$$y'' + 4y = \frac{d^{2}y}{dx^{2}} + 4y$$
		
		$$y'' + 4y = \frac{d^{2}}{dx^{2}} \ \Big[c_{1} sen(2x) + c_{2} cos(2x)\Big] + 4 \ \Big[c_{1} sen(2x) + c_{2} cos(2x)\Big]$$
		
		$$y'' + 4y = \frac{d^{2}}{dx^{2}} \ \Big[c_{1} sen(2x) + c_{2} cos(2x)\Big] + 4 \ c_{1} sen(2x) + 4 \ c_{2} cos(2x)$$
		
		$$y'' + 4y = -4 \ c_{1} sen(2x) - 4 \ c_{2} cos(2x) + 4 \ c_{1} sen(2x) + 4 \ c_{2} cos(2x)$$
		
		$$y'' + 4y = 0$$
		
		Logo, \ $y(x) = c_{1} sen(2x) + c_{2} cos(2x)$ é solução da EDO \ $y''+4y = 0$.
	\end{sol}
	
	
\end{document}